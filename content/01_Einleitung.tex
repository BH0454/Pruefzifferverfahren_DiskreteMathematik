\section{Einleitung}

% Prüfzifferverfahren sind ein essenzielles Werkzeug zur Sicherstellung der Datenintegrität in zahlreichen Anwendungen des täglichen Lebens. Sie dienen dazu, Eingabefehler, wie etwa Tippfehler oder Übertragungsfehler, zu erkennen und somit die Zuverlässigkeit von Daten zu erhöhen.

% In der modernen Informationsverarbeitung spielen Prüfziffern eine zentrale Rolle, beispielsweise bei der Validierung von Bankkontonummern, ISBN-Nummern für Bücher oder internationalen Standardcodes wie der IBAN. Diese Verfahren basieren auf mathematischen Prinzipien der Diskreten Mathematik und ermöglichen eine effiziente Fehlererkennung.

% Im Folgenden werden die Grundlagen der Prüfzifferverfahren erläutert, verschiedene Methoden vorgestellt und deren praktische Anwendungen diskutiert. Ziel ist es, ein tieferes Verständnis für die Funktionsweise und die Bedeutung dieser Verfahren zu vermitteln.